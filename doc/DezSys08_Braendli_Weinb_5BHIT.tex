\documentclass[letterpaper, 12pt]{article}

%%%%%%%%%%%%%%%%%%%%%%%%%%%%%
% DEFINITIONS
% Change those informations
% If you need umlauts you have to escape them, e.g. for an ü you have to write \"u
\gdef\mytitle{Protokoll}
\gdef\mythema{DezSys08 - GPGPU}

\gdef\mysubject{Systemtechnik-DezSys}
\gdef\mycourse{5BHIT 2015/16}
\gdef\myauthor{Michael Weinberger}

\gdef\myversion{0.1}
\gdef\mybegin{15. Januar 2016}
\gdef\myfinish{23. Januar 2016}

\gdef\mygrade{Note:}
\gdef\myteacher{Betreuer: Borko/Micheler}
%
%%%%%%%%%%%%%%%%%%%%%%%%%%%%%

\input special/preamble.tex

\let\tempsection\section
\renewcommand\section[1]{\vspace{-0.3cm}\tempsection{#1}\vspace{-0.3cm}}
\WithSuffix\newcommand\section*[1]{\tempsection*{#1}}

\let\tempsubsection\subsection
\renewcommand\subsection[1]{\vspace{0cm}\tempsubsection{#1}\vspace{0cm}}

\let\tempsubsubsection\subsubsection
\renewcommand\subsubsection[1]{\vspace{0cm}\tempsubsubsection{#1}\vspace{0cm}}

\linespread{0.94}

\lhead{\mysubject}
\chead{}
\rhead{\bfseries\mythema}
\lfoot{\mycourse}
\cfoot{\thepage}
% Creative Commons license BY
% http://creativecommons.org/licenses/?lang=de
\rfoot{\ccby\hspace{2mm}\myauthor}
\renewcommand{\headrulewidth}{0.4pt}
\renewcommand{\footrulewidth}{0.4pt}

\begin{document}
\parindent 0pt
\parskip 6pt

\pagenumbering{Roman} 
\input{special/title}

\clearpage
\thispagestyle{empty}
\tableofcontents

\newpage
\pagenumbering{arabic}
\pagestyle{fancy}

%\vspace{-0.5cm}
\section{Einführung}
\subsection{Ziele}
Die Aufgabe beinhaltet eine Recherche über grundsätzliche Einsatzmöglichkeiten für GPGPU. Dabei soll die Sinnhaftigkeit der Technologie unterstrichen werden. Die Fragestellungen sollen entsprechend mit Argumenten untermauert werden.
Im zweiten Teil der Arbeit soll der praktische Einsatz von OpenCL trainiert werden. Diese können anhand von bestehenden Codeexamples durchgeführt werden. Dabei wird auf eine sprechende Gegenüberstellung (Benchmark) Wert gelegt.
Die Aufgabenstellung soll in einer Zweiergruppe bearbeitet werden.

\subsection{Aufgabenstellung}
Informieren Sie sich über die Möglichkeiten der Nutzung von GPUs in normalen Desktop-Anwendungen. Zeigen Sie dazu im Gegensatz den Vorteil der GPUs in rechenintensiven Implementierungen auf [1Pkt]. \\
Gibt es Entwicklungsumgebungen und in welchen Programmiersprachen kann man diese nutzen [1Pkt]? \\
Können bestehende Programme (C/C++ und Java) auf GPUs genutzt werden und was sind dabei die Grundvoraussetzungen dafür [1Pkt]? \\ Gibt es transcompiler und wie kommen diese zum Einsatz [1Pkt]? \\

Präsentieren Sie an einem praktischen Beispiel den Nutzen dieser Technologie. Wählen Sie zwei rechenintensive Algorithmen (z.B. Faktorisierung) und zeigen Sie in einem aussagekräftigen Benchmark welche Vorteile der Einsatz der vorhandenen GPU Hardware gegenüber dem Ausführen auf einer CPU bringt (OpenCL). \\  \\ Punkteschlüssel: \\

Auswahl und Argumentation der zwei rechenintensiven Algorithmen (Speicher, Zugriff, Rechenoperationen) [0..4Pkt] \\
Sinnvolle Gegenüberstellung von CPU und GPU im Benchmark [0..2Pkt] \\
Anzahl der Durchläufe [0..2Pkt] \\
Informationen bei Benchmark [0..2Pkt] \\
Beschreibung und Bereitstellung des Beispiels (Ausführbarkeit) [0..2Pkt] \\

\subsection{Links}
OpenCL-Examples von René Hollander \& Paul Kalauner \cite{opencl_example}

\clearpage

\section{Möglichkeiten GPU Desktopanwendungen, Vorteil GPU bei rechenintensiven Implementierungen}
GPGPU (General Purpose Computation on Graphics Processing Unit) ist eine Programmier-Schnittstelle mit der Möglichkeit, allgemeine Berechnungen von der GPU ausführen zu lassen. Die ursprüngliche Aufgabe von Grafikchips ist, den Bildschirm mit Pixel zu füllen. Inzwischen eignen sich die umfangreichen Befehlssätze der GPUs auch für allgemeine Berechnungen, dies ist ein vergleichsweise neuer Trend in der Computertechnik, der darauf abzielt freie Rechenleistung konsequent auszunutzen. \\
Eine CPU besitzt wenige, im Serverbereich bis zu 16, Kerne die für jedmögliche Art von Berechnungen ausgelegt sind. Eine GPU besitzt mehrere Hundert, heute sogar schon bis zu über 1536 Kerne die parallel arbeiten. Ein Vorteil, bestimmte komplexe Berechnungen hierauf auszulagern erscheint klar sichtbar. 
\begin{figure}[!h]
	\begin{center}
		\includegraphics[width=1\linewidth]{images/floating-point-operations-per-second}
		\caption{GPU vs CPU, reine Rechenleistung \cite{comparison}}
		\label{Angabe}
	\end{center}
\end{figure}
Bei der reinen Rechenleistung ist wie auf dem Diagramm ersichtlich die GPU weit vorne. Die Rechenleistung ist in FLOPS (Floating Point Operations), die theoretisch mögliche Spitzenleistung, angegeben.

GPGPU wird verwendet um Rechenaufwendige Applikationen zu beschleunigen. Besonders aufwendig sind vor allem die folgenden Anwendungsbereiche Simulationswissenschaften, Medizin, Deep Learning. \\

Grafikkarten sind auf massive Parallelität ausgelegt, dies stellt aber auch Entwickler vor neue Herausforderungen. Es können je nach Anzahl an programmierbaren Shadereinheiten über 1000 Berechnungen zugleich in einem Takt ausgeführt werden. \cite{elko} \cite{Ausarbeitung}

\section{Entwicklungsumgebungen, Programmiersprachen}
asda
\newpage

\section{Bestehende Programme (C/C++ und Java) auf GPUs nutzen und Grundvoraussetzungen}
adf
\newpage

\section{Transcompiler und deren Einsatz}
asdf
\newpage

\section{Praktisches Beispiel: Benchmark}
asdf
\subsection{Auswahl und Argumentation der Algorithmen}
asdf
\subsection{Gegenüberstellung CPU/GPU}
sadf
\subsection{Anzahl der Durchläufe}
sadf
\subsection{Informationen bei Benchmark}
asdf
\subsection{Beschreibung, Bereitstellung des Beispiels}
sadf

\newpage

\bibliographystyle{unsrt}
\bibliography{references}
\lstlistoflistings
\listoffigures

\end{document}
